\chapter{可読性の向上に関する評価}
\label{readability}

本章では、~\ref{implementation}章で説明したTreeSwiftと現行のSwiftコンパイラとの比較評価の方法とその結果について述べる。


\section{評価概要}

評価の目的はSelf-host化によってSwiftコンパイラの可読性が向上していることを検証することである。

\subsection{比較対象}

本研究では、各コンパイラの構文解析を行う箇所にのみ注目して比較評価を行う。

比較に用いるコンパイラの要素としては他に意味解析部分、コード生成部分、最適化部分が考えられるが、これらは次の理由により本研究の目的に見合った単純な比較ができないため対象としない。

まず、コード生成部分については現行のSwiftコンパイラとTreeSwiftとの間で使用するLLVMのAPIが異なってしまう点で問題がある。
現行のSwiftコンパイラはC++で記述しているためにC++用のLLVM APIを使用しているが、実装するコンパイラではSwiftにC++との相互運用性がないため、C用のLLVM APIを使用している。
これらのAPI間ではその設計が大きく異なり、その差によって結果に大きな影響が出てしまう可能性があるため、コード生成部分は比較の対象としない。

次に最適化部分については、その実装が言語の仕様によって制限されないという点に問題がある。
仕様に制限されないために各コンパイラでのサポート状況によって大きくその実装が異なってきてしまい、同等の機能が比較されていることの確認が難しいため、ここでは比較の対象としない。

最後に、意味解析部分には主に型推論、型検査、型決定後の参照解決、詳細なエラー検出、言語レベルでの最適化などが含まれるが、このステップ中の各項目は密接に関わりあっており、コンパイラから切り出した際の単位が大きくなってしまうという問題がある。
評価する対象の単位が大きいと、その評価の結果は複数の機能間で相互に打ち消し合い、実際には各機能によっていい結果と悪い結果が混在しているにも関わらず、押し並べた結果のみが表出してしまう可能性がある。
そのため、意味解析部分についても比較の対象とはしない。

\subsection{評価方法}

評価は各コンパイラが特定のSwiftプログラムを構文解析する際に実行したコードの行数を比較することで行う。

より具体的には、評価に用いるSwiftプログラムをコンパイラがロードし、字句解析並びに構文解析を行って、メモリ内にそのASTを構築するまでに実行された機械語に対応するソースコードをコンパイラの全ソースコードから抽出し、その行数が少ない方をより可読性が高いとする。
この際、コンパイラオプションの解析などのコンパイルのための下準備を行っている箇所は対象としない。
また、標準ライブラリや依存モジュールの構文解析については、TreeSwiftでは特に独自の形式を定義していないためモジュールの定義にテキストファイルを使用しているのに対して、現行のSwiftでは独自のファイル形式を使用しており、そこで大きな差が生まれてしまうことを避けるために対象としない。

本研究で実装したTreeSwiftは、~\ref{implementation}章で示したとおり現行のSwiftコンパイラと同様の手法で構文解析器が実現されている。
そのため、実行されたコードの行数を比較することで、同じSwiftプログラムを同様に解析できる構文解析器の内容を把握するために参照しなければならない行が少なく、より可読性が高いと考えることができる。

\section{計測}

\subsection{計測内容}

計測には

\begin{itemize}
    \item 実用的な複数のベンチマークプログラム
    \item 構文解析器中の実行可能なコードの90\%以上を実行する1つのプログラム
    \item 構文解析器中の実行可能なコードの90\%以上を実行する複数のプログラム
\end{itemize}

を使用する。
また、計測がSwiftの構文を充分に網羅していることを示すため、各プログラムのコンパイル時に実行されたソースコードの、現行のSwiftコンパイラの構文解析器中の実行可能なコード全体に占める割合も計算する。

\subsection{計測結果}

\begin{itemize}
    \item 各プログラムで行数は少なくなる
    \item 実行されたコードの割合についても記述
\end{itemize}

\section{考察}
\label{readability:discussion}

\begin{itemize}
    \item コードの割合から充分に網羅されていることを確認
    \item 行数が少なくなっているので可読性の向上はできている
    \item 行数が少なくなっている原因を幾つかの箇所をピックアップして考察する
    \item 行数が少なくなっている原因の1つとして、コンパイラのチューニングが甘いという可能性がある(C++ではポインタやインライン関数など様々な機能でチューニングできる)
    \item Self-host化はコンパイラのコード全体に影響をあたえるので、他の影響が出ていないかを調べる必要がある
    \item 他の言語の事例を上げて性能以外のデメリットについてもまとめて問題がないか確認する
\end{itemize}

%%% Local Variables:
%%% mode: japanese-latex
%%% TeX-master: "../thesis"
%%% End:
