\chapter{結論}
\label{conclusion}

本章では、本研究の結論と今後の展望を示す。

\section{本研究の結論}

本研究では、Bootstrap化によってSwiftコンパイラの可読性を向上することを目的として、実際にSwiftで記述したSwiftコンパイラを実装し、その構文解析器について現行のコンパイラとの比較を行った。
実装したコンパイラの構文解析器はソースコードの行数から可読性の変化を判断するために、現行のコンパイラと同様の手法で設計した上で同等の機能を有していることを示した。

ソースコードの行数を比較した結果から、SwiftコンパイラをBootstrapすることによりその可読性を向上できる可能性が高いことが分かった。

ただし、可読性の検証に用いた構文解析器で性能の評価も行うことにより、Bootstrapしたコンパイラが現行のコンパイラと全く同じコード生成機能を有したと仮定すると、Bootstrapによって可読性の向上と共にコンパイルの実行速度の低下とコンパイル中の最大使用メモリ量の増加を招く可能性が高いことも分かった。

これらの結果に加えて~\ref{explain-bootstrap:issue}節で事例から検討したSwiftのBootstrapにおける課題を考慮にいれると、Bootstrapによって可読性が向上するからといってすぐにそれを適用するのではなく、本当に現在のSwiftに必要な機能を今後の方向性から検討しなおし、実際の適用にあたっては慎重に判断していく必要があると考えられる。

\section{今後の展望}

\subsection{構文解析器以外の比較}

\subsection{継続的な比較}

%%% Local Variables:
%%% mode: japanese-latex
%%% TeX-master: "../thesis"
%%% End:
