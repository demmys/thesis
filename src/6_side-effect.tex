\chapter{事例からみるSelf-host化の副作用}
\label{side-effect}

~\ref{readability:discussion}節で述べたように、Self-host化はコンパイラに対して可読性の向上以外の様々な影響を与えている可能性がある。
そこで本章では、これまでにBootstrapを行ってきた高級汎用言語の事例を紹介し、それらの例からSwiftのSelf-host化において発生しうる可読性の向上以外の作用について整理する。

\section{Bootstrapの事例}
\label{explain-bootstrap:instance}

BootstrapはFortranやLispのような比較的古い言語からGoやF\#のような比較的新しい言語まであらゆる時代の言語で行われており、その目的は様々である。
本節では、その中から特に近年よく使用されており、先に他の言語による実装が十分な機能を持ってリリースされているにもかかわらずBootstrapを行った高級汎用言語である、Goのgo、PythonのPyPy、C\#の.NET Compiler Platformの3つの事例について紹介する。
なお、本節でSelf-host化ではなくBootstrap化を行った言語を対象としているのは、~\ref{introduction:background}節で述べたのと同様にSelf-host化だけを行っている言語にはコンパイラの大部分を書き換えているものと一部分だけを書き換えているものだけがあり、千疋が難しいためである。

\subsection{Go - go}
\label{explain-bootstrap:instance:go}

Goは2009年にGoogle社より発表された、構文の簡潔さと効率の高さ、並列処理のサポートを中心的な特徴とする静的型付コンパイラ型言語である。
発表から6年を経た2015年にリリースされたバージョン1.5でBootstrapが行われ、それまでCで記述されていたコンパイラは完全にGoへと書き換わった。

\subsubsection{Bootstrapの目的}

GoコンパイラのBootstrapの目的はCとGoの比較という形で~\cite{go-compiler-overhaul}内で以下のように述べられている。

\begin{itemize}
\item It is easier to write correct Go code than to write correct C code.
\item It is easier to debug incorrect Go code than to debug incorrect C code.
\item Work on a Go compiler necessarily requires a good understanding of Go. Implementing the compiler in C adds an unnecessary second requirement.
\item Go makes parallel execution trivial compared to C.
\item Go has better standard support than C for modularity, for automated rewriting, for unit testing, and for profiling.
\item Go is much more fun to use than C.
\end{itemize}

主にGoを用いたコンパイラの開発がCを用いた場合よりも正確かつ楽になるという点を強調していることから、GoコンパイラのBootstrapの目的は主にコンパイラ開発フローの改善にあったということができるだろう。

\subsubsection{Bootstrapの方法}

Goコンパイラにおいては、~\cite{go-compiler-overhaul}に詳細なBootstrapのプロセスが記述されている。
これによれば、Bootstrapはおおまかに以下の流れで行われた。

\begin{enumerate}
\item CからGoへのコードの自動変換器を作成する。
\item 自動変換機をCで書かれたGoコンパイラに対して使用し、新しいコンパイラとする
\item 新しいGoで書かれたコンパイラをGoにとって最適な記法へと修正する
\item プロファイラの解析結果などを用いてGoで書かれたコンパイラを最適化する
\item コンパイラのフロントエンドをGoで独自に開発されているものへと変更する
\end{enumerate}

この手法では、特にCからGoへの自動変換器を作成したことで、Cで書かれたコンパイラの開発を止めること無くGoへの変換の準備を行うことができた、という点が優れている。
ただしこの手法を取れたのは、GoがCに近い機能を多く採用していたこと、CとGoが共に他の高級汎用言語と比べて少ない構文しか持っていなかったことに依るところが大きい。

また、バージョン1.5以降のGoコンパイラにおいてはまずGoのバージョン1.4を用いてコンパイルし、その後にそのコンパイラを再度自分自身でビルドすることによって最新バージョンのコンパイラでコンパイルした最新バージョンのコンパイラを得る~\cite{go-bootstrap-plan}。
この多少複雑な形によりバージョン1.5以降でもコンパイラをCから独立させることができるが、その代わりにGoのバージョン1.4に依存し続ける点には注意しなくてはならない。

\subsubsection{Bootstrapの結果}

Bootstrapが行われた結果、Goコンパイラのコンパイル速度が低下したことがバージョン1.5のリリースノート~\cite{go-15-release}で言及されている。
これについて同リリースノートではCからGoへのコード変換がGoの性能を十分に引き出せないコードへの変換を行っているためだとしており、プロファイラの解析結果などを用いた最適化が続けられている。


\subsection{Python - PyPy}
\label{explain-bootstrap:instance:python}

Pythonは1991年に発表されたマルチパラダイムの動的型付けインタプリタ型言語である。
Pythonの最も有名な実装であるCPythonはC言語で書かれているが、そのCPythonと互換性があり、Bootstrapされた全く別のコンパイラが2007年にPyPyという名前でリリースされている。

このPyPyではCPythonと比べてJITコンパイル機能を備えている点が最も大きな違いとなっている。

\subsubsection{Bootstrapの目的}

PyPyがBootstrapを行った目的はそのドキュメントである~\cite{pypy-doc}内の以下の記述から、特にPythonという言語の持つ柔軟性と、それによる拡張性の高さを利用するためであると読み取れる。

\begin{quotation}
This Python implementation is written in RPython as a relatively simple interpreter, in some respects easier to understand than CPython, the C reference implementation of Python. We are using its high level and flexibility to quickly experiment with features or implementation techniques in ways that would, in a traditional approach, require pervasive changes to the source code.
\end{quotation}


\subsubsection{Bootstrapの方法}

PyPyのインタプリタは、PyPyと同時に開発されているRPythonというPythonのサブセット言語で実装されており、RPythonはRPythonで書かれたプログラムをCなどのより低レベルな言語に変換する役割を担う~\cite{rpython-doc}。

そのため、RPythonの実行時の性能はPyPy自体の性能に一切関与せず、例えばPythonで記述されているRPythonがPyPyで実行されているかCPythonで実行されているかはPyPyの性能に何ら影響を与えない。

このRPythonという変換器による仲介と、既存実装であるCPythonとの互換性がPyPyのBootstrapを可能にしている。


\subsubsection{Bootstrapの結果}

PyPyについては多くのベンチマークにおいて互換性のあるCPythonのバージョンに対してその実行速度が向上していることが示されている~\cite{speed-pypy-org}。
これはPyPyが単にBootstrapを行っただけでなく、RPythonによってPyPyインタプリタをネイティブコードにコンパイルできるよう仲介した上で、JITコンパイル機能を付加したためである。

このように、Bootstrapを行ったインタプリタを直接そのインタプリタで実行するのではなく、より高速に動作する形へ変換して実行することで、性能の低下を免れられ、それどころか独自の拡張によってそれまでの実装よりもより高い性能を得られる場合がある。
ただし、この手法を取った場合はPyPyにおけるCのような他の低級言語への依存が残ってしまい、その可搬性に制限が生じてしまう可能性があるという点には注意する必要がある。


\subsection{C\# - .NET Compiler Platform}
\label{explain-bootstrap:instance:csharp}

C\#は2000年にMicrosoft社より.NET Frameworkを利用するアプリケーションの開発用に発表された、マルチパラダイムの静的型付けコンパイラ型言語である。
2014年に同社はC++で記述されていたコンパイラのBootstrapを行い、Visual Basic .NET と合わせてコンパイラ中の各モジュールをAPIによって外部から利用できるようにした.NET Compiler Platformをプレビュー版としてリリースした。
その後2015年にはVisual Studio 2015における標準のC\#コンパイラとして.NET Compiler Platformを採用するようになっている。

\subsubsection{Bootstrapの目的}

.NET Compiler Platformはコンパイラの構文解析や参照解決、フロー解析などの各ステップを独立したAPIとして提供している~\cite{roslyn-doc}。
これにより、例えばVisual StudioなどのIDEはこれらのAPIを使用することで、いちからC\#のパーサを構築すること無くシンタックスハイライトや定義箇所の参照機能を提供できる。
そうしたライブラリ的機能をスムーズに利用できるようにするためには、.NET Compiler Platform自体がそれを利用するVisual Studioの拡張などと同様の言語で提供されている必要があった。

その結果、.NET Compiler PlatformはC\#コンパイラをC\#、Visual Basic .NETをVisual Basic .NETで記述するBootstrapの形式で開発することとなっている。

\subsubsection{Bootstrapの方法}

.NET Compiler PlatformはVisual Studio 2013以前に使用されていたVisual C\#とは独立して開発され、Visual Studio 2013に採用されていたC\# 5の次期バージョンであるC\# 6の実装となっていた。
そのため、.NET Compiler Platformの開発においてVisual C\#に対する大きな変更などは行われておらず、全ての新機能を.NET Compiler Platformのみに対して適用するだけで事足りている。

また、.NET Compiler Platformの最新版はVisual Studioの最新版とともにバイナリ形式で配布されることが前提となっており、公開されているソースコードからビルドを行う場合でも配布されているコンパイラを使用する。
そのため、配布されているVisual Studioが実行可能なプラットフォーム以外で.NET Compiler Platformを使用するためにはそれを実行可能な環境でクロスコンパイルするか.NET Compiler Platform以外のC\#コンパイラを用いてコンパイルする以外に方法がないが、その明確な手立ては示されていない~\cite{roslyn-cross-platform}。

\subsubsection{Bootstrapの結果}

Microsoft社ではBootstrapを行うにあたってその性能に対して非常に注力しており、結果としてBootstrap後も想定していた充分によい性能が発揮できていると~\cite{roslyn-performance}内で述べている。


\section{SwiftにおけるSelf-host化の副作用}
\label{side-effect:swift}

表~\ref{table:bootstrap-merit}に~\ref{explain-bootstrap:instance}節で述べた事例からBootstrapにおける利点をまとめ、SwiftのBootstrap時にそれらの利点が得られる可能性があるか否かをまとめた。

\begin{table}[hb]
    \begin{center}
        \caption{Swiftでも享受できる可能性のあるBootstrapの利点}
        \begin{tabular}{|m{10cm}|c|}
            \hline
            利点 & Swiftでの享受 \\
            \hline
            プログラムの記述やデバッグがより容易になる & ◯ \\
            \hline
            コンパイラの開発を行うために必要な知識が減る & ◯ \\
            \hline
            並列化やモジュール化、テスト、プロファイリングなどにおいて高いサポートが得られる & × \\
            \hline
            より柔軟で拡張性が高くなる & ? \\
            \hline
            コンパイラの各フローをライブラリとして提供できる & × \\
            \hline
        \end{tabular}
        \label{table:bootstrap-merit}
    \end{center}
\end{table}

Swiftで記述されたプログラムは現行のコンパイラで使用されているC++で記述されたものと比較して、記述やデバッグが容易になる可能性が高い。
詳細は~\ref{explain-swift}章で述べているが、メモリの管理を自動的に行うARC方式や強力な型推論と型検査はC++で必要となる冗長な記述を省略し、多くのエラーをコンパイル時に発見する。
ただし、飽くまでもこれはプログラムの記述の容易さだけに関する議論であり、~\ref{explain-bootstrap:instance:go}節で見たように、より簡単な記述で同等の性能が得られるとは限らないという点に注意されたい。

コンパイラの開発を行うために必要な知識が減るという点は~\ref{explain-bootstrap:instance:python}節で述べたPyPyのような形でなければほとんど全てのBootstrapに対して有効である。
Swiftはコンパイラ型言語であるため、よほど特殊な方法を採用しない限りはこのメリットを享受できると考えられる。

並列化やモジュール化、テスト、プロファイリングに対するサポートはSwiftとC++の間で同等か、Swiftは特にMac OS X以外のプラットフォームにおける各機能のサポートが不十分なため、C++の方に分配が上がる可能性が高い。

~\ref{introduction:background}節でも述べたように、言語の柔軟性および拡張性にはその可読性が大きく影響することが~\cite{elshoff}によって示されているため、これがSwiftにおいても利点となるかどうかは可読性が向上するかどうかという本研究の評価によって判断することができる。

~\ref{explain-bootstrap:instance:csharp}節で挙げた.NET Compiler Platformのようにコンパイラの各フローをライブラリとして提供することは設計次第で可能だろう。
しかし、現状のSwiftにはC\#に対するVisual Studioのようなその言語で記述されたIDEや言語のためのツールなどは存在しておらず、そのAPIをSwiftで提供したとしても、特筆すべきほどの利点にはなり得ない可能性が高い。

表~\ref{table:bootstrap-demerit}に~\ref{explain-bootstrap:instance}節で述べた事例からBootstrapにおいて発生しうる課題をまとめ、SwiftのBootstrap時にそれらの課題が問題となるかどうかをまとめた。

\begin{table}[hb]
    \begin{center}
        \caption{SwiftのBootstrap時に発生しうる課題}
        \begin{tabular}{|m{10cm}|M{3cm}|}
            \hline
            課題 & SwiftのBootstrap時に問題となるか \\
            \hline
            現行のコンパイラの開発に対する影響 & ◯ \\
            \hline
            過去のバージョンに対する依存 & ◯ \\
            \hline
            コンパイル速度の低下 & ? \\
            \hline
            他の低級言語への依存 & × \\
            \hline
            他のプラットフォームへの移植の煩雑化 & ◯ \\
            \hline
        \end{tabular}
        \label{table:bootstrap-demerit}
    \end{center}
\end{table}

Swiftの言語仕様は日々メーリングリストで改善のための議論が行われており、既に次期バージョンである3.0での破壊的な変更も定まっているように、まだまだ安定していない。
そのため、~\ref{explain-bootstrap:instance:csharp}節で挙げたC\#のように全く別のプロジェクトとしてBootstrapされたSwiftコンパイラを作成する場合には、現行のコンパイラとBootstrapしているコンパイラの両方のメンテナンスコストが非常に高くなってしまう。
これを防ぐためにはBootstrapのプロセスにおいてコンパイラへの大きな仕様変更などを行わないようにしなくてはならず、現行のコンパイラの開発に対する影響は避けられなくなってしまう。
なお、~\ref{explain-bootstrap:instance:go}節で述べたGoの例のようにC++からSwiftへの変換器を作成する形式は現実的ではない。
なぜならば、C++とSwiftは互いに言語仕様が複雑かつ多様であり、かつSwiftの言語仕様は先述の通り破壊的に変更されているため、その変換器を作成するためのコストはBootstrapによって得られるメリットよりも格段に大きくなる可能性が高いからである。

過去のバージョンに対する依存は、~\ref{explain-bootstrap:instance:go}節で述べたGoのような方法を取れば避けられない課題である。
かつ、現状のSwiftにおいてはバージョン間で破壊的な変更があるため、それがコンパイラの機能を大きく制限してしまう可能性が高い。
この問題は~\ref{explain-bootstrap:instance:csharp}節で述べた.NET Compiler Platformのようにコンパイラの配布の基本をバイナリ形式とすることで解決できる。
ただし、その場合は.NET Compiler Platformと同様に他のプラットフォームへの移植の煩雑化という問題とのトレードオフに陥る点には注意しなくてはならない。

コンパイル速度の低下は~\ref{explain-bootstrap:instance:go}節にあるGoのように変換器を使用しなかったとしても起こりうる可能性があるが、~\ref{explain-bootstrap:instance:csharp}節にある.NET Compiler Platformの事例のようにそのパフォーマンス改善に注力することで十分な性能を得られる可能性も等しくある。
この結果を形式的な議論から導くことは難しいが、本研究で可読性の評価に用いる構文解析器は~\ref{introduction:purpose}節で示した構文解析器の比較が可読性が向上したかどうかの検証に使用できるということと同じ理由で、その実行速度やメモリ使用量を比較し、その変化を判断するために使用できる。

他の低級言語への依存は、~\ref{explain-bootstrap:merit}節でも述べたようにSwift自体がコンパイラ型言語であるため、起こりづらいと考えられる。

\section{性能の差の検証方法}

%%% Local Variables:
%%% mode: japanese-latex
%%% TeX-master: "../bthesis"
%%% End:
