\chapter{結論}
\label{conclusion}

本章では、本研究のまとめと今後の展望を示す。

\section{本研究のまとめ}

本研究では、現行のC++で記述されたSwiftコンパイラをBootstrapすることによって利益を得られるかどうかを判断することを目的として、実際にSwiftで記述したSwiftコンパイラを実装し、コンパイラの基幹機能である構文解析器について現行のコンパイラとの比較を行った。
実装したコンパイラの構文解析器はソースコードの行数から柔軟性と拡張性の変化を判断し、実行速度と実行時のメモリ使用量から性能の変化を判断するために、現行のコンパイラと同様の手法で設計した上で、同等の機能を有していることを示した。

ソースコードの行数、実行速度、実行時のメモリ使用量を比較した結果から、SwiftでSwiftコンパイラを記述することによりコンパイラの柔軟性と拡張性を向上できるが、性能については下がる可能性があり、他の利点や課題を考慮にいれると現時点におけるSwiftコンパイラについてはBootstrapによって十分な利益が得られない可能性が高いことを示した。


\section{今後の展望}

\subsection{構文解析器以外の比較}

\subsection{継続的な比較}

%%% Local Variables:
%%% mode: japanese-latex
%%% TeX-master: "../thesis"
%%% End:
