\chapter{序論}
\label{introduction}

\section{背景}
\label{introduction:background}

\subsection{Swiftのオープンソース化による変化}

2015年12月、Apple社が予てよりCocoaおよびCocoa Touchフレームワークを用いたソフトウェアの開発用プログラミング言語として提供していたSwiftのコンパイラをオープンソース化した。
このプロジェクトのオープンソース化は、単にソースコードを公開することだけを指すのではない。
Open Source Initiativeが定義しているように、ソースコードを公開した上でそのソフトウェアの改変版や派生版の販売・頒布を許可するのが一般的なオープンソースプロジェクトのスタイルである~\cite{opensource}。
しかし、だからといって誰もが元々のプロジェクトと全く異なる場所で独自に拡張や修正をしているだけではソフトウェアの発展につながりづらい。
そのため、オープンソース化は単にソースコードの管理方法を変えるだけでなく、ソフトウェアの開発フローそのものの変更を伴う場合が多い。

クローズドプロジェクトではプロジェクトに関わる開発者の人数がはっきりしているため、そのリソースに見合った明確な開発方針がマネージャーによって決定され、それに従って役割分担された人員によって実装やレビューが進められる。
それに対してオープンソースプロジェクトでは、プロジェクトに関わる開発者やユーザから成るコミュニティでの議論を通して現在のプロジェクトで行うべき修正や拡張が提案・検討され、それに対して必要性を感じた開発者が自発的に実装やレビューを行う。
すなわち、クローズドプロジェクトではそのプロジェクトに割ける人員やその時間といったリソースが開発を進めるための最も重要な因子となっているのに対して、オープンソースプロジェクトではその開発に必要な労力を投じるに値する充分な興味・関心が開発者にあれば、必要なリソースは自然と割り当てられるようになる~\cite{raymond}。

\subsection{Swiftのソースコード可読性における伸びしろ}

Swiftコンパイラについて言えば、そのプロジェクト自体に対する開発者の関心は非常に高いといえる。
2001年以来10年以上にわたってApple社が提供するCocoaフレームワークのコア言語となっていたObjective-C~\cite{objective-c}を完全に置き換える言語として発表されたSwiftはObjective-Cの膨大なユーザをそのまま受け継ぐ形となり、インターネット上の情報量を元にしたプログラミング言語の人気度格付けにおいても過去に類を見ない早さで上位にランキングされるようになっっているからだ~\cite{tiobe}~\cite{redmonk}。

しかし一方で、その開発に必要な労力についてはまだ改善の余地が残されている。
ソフトウェアの修正や拡張といったメンテナンスのためのコストには、ソフトウェアのソースコードの可読性が大きな影響力を持つことがElshoffやBankerによる~\cite{elshoff}~\cite{banker-datar}~\cite{banker-davis}などで言及されているが、Swiftコンパイラにおいてそのソースコードの可読性は既存コードのコーディングスタイルへの追従やレビューの徹底などによってのみ保たれている。
このような方法では可読性が向上していく望みが薄いだけでなく、新しく修正・追加されたコードがオープンソース化以前のコードの両を上回ったり、レビュアーが多様化することによって維持することすらできなくなる。

つまり、Swiftプロジェクトをさらに活発なものとするためには、Swiftコンパイラのソースコード可読性を何かしらの方法で向上していく必要がある。

\section{本研究が着目する課題}
\label{introduction:issue}

本研究では、~\ref{introduction:background}節で述べたように、Swiftプロジェクトにおいてコンパイラのソースコード可読性を向上する必要があるという課題に着目する。

\section{本論文の構成}

本論文における以降の構成は次の通りである。

~\ref{open-source}章では、本研究の対象とするプログラミング言語Swiftが行ったオープンソース化による影響についてまとめ、本研究の着目する課題について整理する。
~\ref{readability}章では、ソフトウェアのソースコード可読性を向上し評価する方法についてまとめた上で本研究がSwiftのソースコード可読性向上のために取るアプローチであるSelf-host化について説明し、他言語におけるSelf-host化の事例を紹介する。
~\ref{refinement}章では、本研究で実装するSelf-host化されたSwiftコンパイラに必要な設計について述べ、実装したコンパイラの概要について述べる。
~\ref{evaluation}章では、実際に現行のSwiftコンパイラとSelf-host化されたコンパイラの可読性を比較し、その結果について考察する。
~\ref{conclusion}章では本研究のまとめと今後の課題についてまとめる。

%%% Local Variables:
%%% mode: japanese-latex
%%% TeX-master: "../thesis"
%%% End:
