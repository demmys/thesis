\chapter{序論}
\label{introduction}

\section{背景}
\label{introduction:background}

2015年12月、Apple社が予てよりCocoaおよびCocoa Touchフレームワークを用いたソフトウェアの開発用プログラミング言語として提供していたSwiftをオープンソース化し、同時にLinuxを中心としたさまざまなプラットフォーム上でのソフトウェア開発に使用するための拡張を開始した
% TODO 出典
。
これによりSwiftは、Objective-Cの担ってきたiOSやMac OS Xなどの特定プラットフォームに向けたソフトウェアだけでなく、C++やJavaなど他の汎用プログラミング言語が担ってきたソフトウェアの開発においてもそれらの代替となり得る可能性を持つこととなっており、今後はこれまで以上に様々な拡張と修正が行われていくと予想される。
それに加え、オープンソースソフトウェアにおいては多くのプログラマが開発に関わるようになる都合から、拡張や修正のためのコードに対するレビューのプロセスがバグを事前に防ぐためにより重要となる
% TODO 出典
。

このような状況の変化によって、Swiftコンパイラにはこれまで強く求められていなかったある特徴が求められるようになっている。
それは、Swiftコンパイラのソースコード自体における高い可読性である。

プログラムに対する拡張や修正の効率は、そのプログラムの可読性に大きな影響を受けることがElshoffらによる~\cite{elshoff}で言及されている。
また、コードレビューのようなプロセスでは、プログラムにおける実行速度などの性能の高さではなくコードの読みやすさによってその効率が左右されることが明らかである。

これらのことから、以前はSwiftコンパイラについてよく知る極小数のメンバーによって開発が行われていたために顕著化していなかった高い可読性に対する要求が、今後は高まっていくであろうと考えられる。

一方、Swift以外の汎用言語のコンパイラにおいてはそのソースコードの可読性を高めることを目的の内の1つとして、コンパイラをそのコンパイル対象の言語自体で開発するSelf-host化を行っている例がよく見られる。

表~\ref{table:bootstrapping-languages}はWeb検索エンジンにおけるクエリヒット数からプログラミング言語の知名度を格付けしたTIOBE Index~\cite{tiobe}の2015年12月版において上げられている言語の内、汎用言語であるものだけを上位から20言語抽出し、それらの主要なコンパイラにおいてBootstrap化されているものがあるかをまとめたものである。
なお、BootstrapとはSelf-host化によってコンパイラのソースコードから他言語への依存を排除し、以降の新しいバージョンのコンパイラをそのコンパイラ自身でコンパイルできるようにすることを指す。
単にSelf-host化だけが行われている場合はコンパイラ中のコンパイル対象言語で記述された箇所がごく僅かである場合もあるため、表~\ref{table:bootstrapping-languages}ではBootstrap化しているかどうかについてまとめている。

表~\ref{table:bootstrapping-languages}中の20言語の内だけでもBootstrapを採用しているものが8言語あり、 その中に性能の問題からコンパイラ用の言語として採用されづらいインタプリタ型言語なども含まれていることを考慮すれば、かなりの言語がSelf-host化されていることが分かる。

しかし、SwiftのSelf-host化についてSwiftコンパイラのレポジトリ内に記載されているFAQ~\cite{swift-faq}では、Self-host化した際に言語環境を用意するプロセスが煩雑化すること、現時点ではSwiftにコンパイラ開発用の特徴を追加するよりも汎用言語としての特徴追加を優先したいことから、短期的にはSelf-host化を行う予定はないと述べられている。

\begin{table}[hb]
    \begin{center}
        \caption{知名度の高いプログラミング言語のBootstrap状況}
        \begin{tabular}{|c|c|c|c|}
            \hline
            順位 & 言語名 & コンパイラ名 & Bootstrap状況 \\
            \hline
            1 & Java & javac & N (C, C++?) \\
            \hline
            1 & Java & OpenJDK & N (C++, Java) \\
            \hline
            2 & C & gcc & N (C++) \\
            \hline
            2 & C & clang & N (C++) \\
            \hline
            3 & C++ & gcc & Y \\
            \hline
            3 & C++ & clang & Y \\
            \hline
            3 & C++ & Microsoft Visual C++ & Y \\
            \hline
            4 & Python & CPython & N (C) \\
            \hline
            4 & Python & PyPy & Y \\
            \hline
            5 & C\# & Microsoft Visual C\# & N (C++) \\
            \hline
            5 & C\# & .NET Compiler Platform & Y \\
            \hline
            6 & PHP & Zend Engine & N (C) \\
            \hline
            7 & Visual Basic .NET & Visual Studio & N (C++, C\#) \\
            \hline
            7 & Visual Basic .NET & .NET Compiler Platform & Y \\
            \hline
            8 & JavaScript & SpiderMonkey & N (C, C++) \\
            \hline
            8 & JavaScript & V8 & N (C++, JavaScript) \\
            \hline
            9 & Perl & perl & N (C) \\
            \hline
            10 & Ruby & Ruby MRI & N (C) \\
            \hline
            12 & Visual Basic & Visual Studio & N (C++, C\#) \\
            \hline
            13 & Delphi/Object Pascal & Delphi & N (?) \\
            \hline
            13 & Delphi/Object Pascal & Free Pascal & Y \\
            \hline
            14 & Swift & swift & N (C++) \\
            \hline
            15 & Objective-C & clang & N (C++) \\
            \hline
            15 & Objective-C & gcc & N (C++) \\
            \hline
            17 & Pascal & Free Pascal & Y \\
            \hline
            17 & Pascal & GNU Pascal & N (C, Pascal) \\
            \hline
            20 & COBOL & GnuCOBOL & N (C, C++) \\
            \hline
            21 & Ada & GNAT & Y \\
            \hline
            22 & Fortran & GNU Fortran & N (C, C++) \\
            \hline
            22 & Fortran & Absoft Fortran Compiler & ? \\
            \hline
            23 & D & DMD & Y \\
            \hline
            24 & Groovy & groovy & N (Java, Groovy) \\
            \hline
        \end{tabular}
        \label{table:bootstrapping-languages}
    \end{center}
\end{table}


\section{本研究が着目する課題}
\label{introduction:issue}

本研究では、~\ref{introduction:background}節で述べたようにSwiftコンパイラの可読性に対する要求が高まっているという点に着目する。

現在のSwiftコンパイラにおけるコードの可読性は既存コードのコーディングスタイルへの追従やレビューの徹底などによって保たれているが、これはコミュニティベースで修正・追加されたコードがオープンソースとなる以前のコードの量を上回ったり、レビュアーが多様化することによって持続できなくなる
% TODO 出典
。
また、これを持続する単純なアプローチとしてコーディング規約などの策定を行うことも考えられるが、詳細な規約はそれを把握するためだけでも多くの労力を要し、結果として可読性向上のための試みが当初の目的であったはずのプログラムの拡張や修正のコストを増大させてしまう可能性がある
% TODO 出典
。


\section{本研究の目的とアプローチ}
\label{introduction:purpose}

本研究ではSwiftコンパイラの可読性を現在のものよりも向上させることを目的とする。
そのアプローチとして、Swiftで記述したSwiftコンパイラを実装し、そのソースコードの可読性を現行のSwiftコンパイラと比較することで、SwiftコンパイラのSelf-host化が可読性の向上に有効であることを示す。

コンパイラのSelf-host化では、そのコンパイラの対象言語が最初に記述された言語よりも必ず後発のものとなるために各機能についてより高い表現力を持つ場合が多いこと、言語仕様中の一部の機能や構文については実際にその機能や構文を用いて実装することができることから、一般的に可読性が高まると期待できる。

ただし、Swiftにおいては現行のコンパイラが既に多くの機能を持つ高級言語であるC++で記述されているために、その可読性の向上が自明ではない。
そのため、本研究ではSwiftとC++の違いから可読性を決定する要因について慎重に検討し、可読性の比較評価には各コンパイラのソースコードの行数に基づく定量的な指標を使用する。

また、コンパイラのSelf-host化はコンパイラのプログラムを書き換えるために可読性の向上だけでない様々な影響を与えうる。
その点について、本論文ではいくつかの事例と実装したコンパイラにおける性能の評価から考察し、実際にSelf-host化を行うにあたっての注意事項としてまとめる。

% さらにBootstrap化は~\ref{introduction:issue}節で述べた手法と比較して、そのコンパイラで使用するプログラミング言語を変更することによる可読性の向上であるために今後更に異なる言語で書き直されないかぎりはその効用が持続し、かつコーディング規約のような追加の制限を設けないので、その可読性の向上がより単純にプログラムの拡張や修正のコストを減少させると期待することができる。
% また、Bootstrapを行うことによってBootstrap前は必要としていたC++の知識が不要となるため、C++に対して馴染みの浅い一部のプログラマにとっては拡張や修正のコストを減少させられるということも期待できる。
%
% ただし、コンパイラのBootstrap化はコンパイラ全体に対して大きな変更を行うこととなるため、可読性の向上以外にも様々な影響を与える可能性が大いにあることには注意しておかなくてはならない。
% 本論文ではそうした副作用も含めたBootstrap化の現実的な優位性について議論できるよう、他の言語におけるBootstrapの事例から可読性とは別の影響についてまとめ、それらがSwiftのBootstrap化においてどのような働きをしうるかについても考察していく。

\section{本論文の構成}

本論文における以降の構成は次の通りである。

~\ref{explain-swift}章では本研究が着目するプログラミング言語Swiftとそのコンパイラの記述言語C++の差異についてまとめ、それら言語で記述されたプログラムの可読性の差を検証するための方法について説明する。
~\ref{treeswift}章では現行のSwiftコンパイラとの比較対象となるSwiftで記述したSwiftコンパイラ「TreeSwift」の構成について述べ、現行のSwiftコンパイラとの比較を行うために必要な要件を満たしていることを確認する。
~\ref{readability-evaluation}章では現行のSwiftコンパイラとTreeSwiftの構文解析器についてその可読性の差異を比較評価し、その結果について考察する。
~\ref{explain-bootstrap}章では改めてコンパイラのSelf-host化についてSwift以外の言語の事例から可読性の向上以外の影響についてまとめ、それらがSwiftのSelf-host化においてどう作用しうるかを考察する。
~\ref{performance-evaluation}章では~\ref{explain-bootstrap}章で上げる性能の低下というSelf-host化が抱えうる課題について、Swiftの事例において起こりうる影響をSwiftコンパイラとTreeSwiftの構文解析器における性能の比較から考察する。
~\ref{conclusion}章では本研究の結論と今後の展望についてまとめる。

%%% Local Variables:
%%% mode: japanese-latex
%%% TeX-master: "../thesis"
%%% End:
