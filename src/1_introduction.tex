\chapter{序論}
\label{introduction}

\section{背景}
\label{introduction:background}

\subsection{オープンソースプロジェクトにおける開発者のモチベーション向上の必要性}

2015年12月、Apple社がCocoaおよびCocoa Touchフレームワークを用いたソフトウェアの開発用プログラミング言語として提供しているSwiftのコンパイラをオープンソース化した。
このプロジェクトのオープンソース化は、単にソースコードを公開することだけを指すのではない。
Open Source Initiativeが定義~\cite{opensource}しているように、ソースコードを公開した上でそのソフトウェアの改変版や派生版の販売・頒布を許可するのが一般的なオープンソースプロジェクトのスタイルである。
しかし、誰もが元々のプロジェクトと全く異なる場所で独自に拡張や修正をしているだけではソフトウェアの発展が困難である。
そのため、オープンソース化は単にソースコードの管理方法を変えるだけでなく、ソフトウェアの開発フローそのものの変更を伴う場合が多い。

クローズドプロジェクトではプロジェクトに関わる開発者の人数がはっきりしているため、そのリソースに見合った明確な開発方針がマネージャーによって決定され、それに従って役割分担された人員によって実装やレビューが進められる。
それに対してオープンソースプロジェクトでは、プロジェクトに関わる開発者やユーザから成るコミュニティでの議論を通して現在のプロジェクトで行うべき修正や拡張が提案・検討され、それに対して必要性を感じた開発者が自発的に実装やレビューを行う。
すなわち、クローズドプロジェクトではそのプロジェクトに割ける人員やその時間といったリソースが開発を進めるための最も重要な因子となっているのに対して、オープンソースプロジェクトではその開発に必要な労力を差し引いても開発者が十分なモチベーションを持てるようにしていく必要がある~\cite{raymond}。

\subsection{Swiftのソースコード可読性における伸びしろ}

Swiftコンパイラについて言えば、そのプロジェクト自体に対する開発者の関心は非常に高いといえる。
2001年以来10年以上にわたってApple社が提供するCocoaフレームワークのコア言語となっていたObjective-C~\cite{objective-c}を完全に置き換える言語として発表されたSwiftはObjective-Cの膨大なユーザをそのまま受け継ぐ形となり、インターネット上の情報量を元にしたプログラミング言語の人気度格付け~\cite{tiobe}~\cite{redmonk}においても過去に類を見ない早さで上位にランキングされるようになっているからだ。

しかし一方で、開発を行う上でのコストにはまだ改善の余地があり、開発に必要なリソースが開発者の興味・関心に応じて充分に割り当てられるかという点については疑問が残る。
ソフトウェアの修正や拡張といったメンテナンスのためのコストには、ソフトウェアのソースコードの可読性が大きな影響力を持つことがElshoff~\cite{elshoff}やBanker~\cite{banker-datar}~\cite{banker-davis}によって言及されているが、Swiftコンパイラにおいてそのソースコードの可読性は既存コードのコーディングスタイルへの追従やレビューの徹底などによってのみ保たれている。
このような方法では可読性が向上していく可能性が低いだけでなく、新しく修正・追加されたコードがオープンソース化以前のコードの量を上回ったり、レビュアーが多様化することによって可読性を維持することすらできなくなる。

つまり、Swiftプロジェクトをさらに活発なものとするためには、Swiftコンパイラのソースコード可読性を何かしらの方法で向上していく必要がある。

\section{本研究が着目する課題}
\label{introduction:issue}

本研究では、~\ref{introduction:background}節で述べたように、Swiftプロジェクトにおいてコンパイラのソースコード可読性を向上する必要があるという課題に着目する。

もちろん、現在もSwiftプロジェクトでは可読性が低くなってしまっている箇所のリファクタリングなども盛んに行われているが、そうした部分的な修正だけでは、プロジェクト全体の開発コストの削減にはつながりづらい。
そのため、本研究ではSwiftコンパイラ全体における可読性の向上を課題として考える。

\section{本論文の構成}

本論文における以降の構成は次の通りである。

~\ref{issue}章では、Swiftコンパイラの可読性をより具体的に表す数値指標について説明し、本研究の目的を明確化する。
~\ref{approach}章では、本研究においてSwiftコンパイラの可読性を向上するために用いるSelf-host化というアプローチについて説明する。
~\ref{implementation}章では、現行のSwiftコンパイラと比較評価を行うために本研究で実装するSelf-host化されたSwiftコンパイラの設計について説明し、その実装を概説する。
~\ref{evaluation}章では、Self-host化されたSwiftコンパイラの可読性を~\ref{issue}章で述べる指標を用いて現行のSwiftコンパイラのものと比較し、その結果について考察する。
~\ref{conclusion}章では本研究のまとめと今後の課題についてまとめる。

%%% Local Variables:
%%% mode: japanese-latex
%%% TeX-master: "../thesis"
%%% End:
