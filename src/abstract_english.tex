Abstract of Bachelor's Thesis - Academic Year 2015
\begin{center}
\begin{large}
\begin{tabular}{|p{0.97\linewidth}|}
    \hline
    Improvement of the Readability of Swift Compiler's Source Code by Self-hosting\\
    \hline
\end{tabular}
\end{large}
\end{center}

~ \\

When a software project changes from closed to open source, as both the number of joining developers \& the extension/modification of program increase, the readability of the source code becomes an important factor for the success of the project.

In December 2015, the Swift programming language project, which is led by Apple, made its source code public and was faced to face with such a change.
In the current process, the readability of Swift compiler is maintained just by the effort of habitually following the coding style of existing code and its strict review.
Therefore, when the new code grows as a fraction of the software, its readability may become out of control.

On the other hand, there is a technique called self-hosting, which is adopted in many general-purpose high-level programming languages.
Self-hosting is a technique for writing the compiler in its own source programming language.
Although there are multiple reasons to adopt the self-hosting style, the most attractive advantage is that developers can reduce their effort for understanding the software.
When the compiler is self-hosted, its developers are not required to learn another language other than the source language.
Furthermore, as the source language is newer than its compiler's description language in the early years, the self-hosting has an effect to change the compiler's source code less complex with the new and expressive grammars of source language.

In this research, I implemented a new Swift compiler which is written in Swift and compare its parser with the parser of Apple's Swift compiler in order to verify the readability enhancement of Self-hosting.
As the result of this research, we could confirm the positive effect of self-hosting on the readability.
But at the same time, as the effect depends on some characteristics of the  grammar of Swift, we are considering that some other languages whose compilers are written in other high-level languages may not benefit from the effect.

~ \\
Keywords : \\
\underline{1. Compiler},
\underline{2. Self-hosting},
\underline{3. Readability of Program},\\
\underline{4. Swift Programming Language}
\begin{flushright}
Keio University, Faculty of Environment and Information Studies\\
Atsuki Demizu
\end{flushright}
