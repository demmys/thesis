\chapter{結論}
\label{conclusion}

本章では、本論文の結論と今後の展望を示す。

\section{本研究の結論}

本研究では、Swiftコンパイラのソースコード可読性をSelf-host化によって向上させることを目的として、Self-host化したSwiftコンパイラを実装し、現行のコンパイラとその可読性を比較した。
両コンパイラの構文解析器について、Swiftのチュートリアル中で用いられている7つのサンプルプログラムを解析するために必要な箇所を抜き出してその行数を比較した結果、特にSwiftの持つパターンマッチなどの機能によってSelf-host化したコンパイラの方が6つのプログラムにおいてより高い可読性を持っていることが示された。
このことから、Self-host化によってSwiftコンパイラのソースコード可読性の向上が充分に期待できるが、その可読性の向上はSwiftの構文が主要因となっているため、他の既に高級言語でコンパイラが記述されている言語においては同様の手法を用いたとしても同じような結果を得られるとは限らない。

\section{今後の展望}

\subsection{構文解析器以外の比較}

構文解析以降の処理ステップについては、その処理が1つ前のステップの影響を受けるため、~\ref{evaluation:measure}節でASTファイル群について可読性が向上しているかどうかを判断できなかったように、本研究の手法ではうまく可読性の比較を行えない可能性が高い。

しかし、コンパイラの他の処理ステップには~\ref{refinement:structure}節で述べたように構文解析器とは異なる様々なアルゴリズムやデータ構造が使用されており、その中には構文解析器のようにSwiftの構文による可読性の向上が充分に発揮されない箇所が存在する可能性もある。

そのため、実際にSwiftコンパイラの全体をSelf-host化するか否かを判断するためには、他の処理ステップについても何らかの手法を用いて評価を行っていく必要があるだろう。

\subsection{継続的な比較}

Swiftは現在も破壊的な変更を含む活発な仕様変更の相次ぐ言語であり、今後のバージョンの仕様で比較した場合は、可読性についての評価結果も変化する可能性がある。
そのため、確かにSwiftに対してSelf-host化が十分な可読性の向上をもたらすことを判断するためには、今後のバージョンでも継続的に同様の比較を行い、その変化を見る必要がある。

%%% Local Variables:
%%% mode: japanese-latex
%%% TeX-master: "../thesis"
%%% End:
