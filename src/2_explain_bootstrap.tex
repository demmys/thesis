\chapter{コンパイラのBootstrap}
\label{explain-bootstrap}

本章では、これまでにBootstrapを行ってきた高級汎用言語の事例を紹介し、それらの例からBootstrapにおける利点と課題について整理する。

\section{Bootstrapの事例}
\label{explain-bootstrap:instance}

近年Bootstrapを行った言語の中にはその際の目的や記述が資料として残っているものがある。
ここではその中から異なる特徴を持つ3つの言語について注目し、各事例についてまとめる。

\subsection{Go}

\subsection{Python}

\subsection{C\#}

\section{Bootstrapの利点}
\label{explain-bootstrap:merit}

\section{Bootstrapの課題}
\label{explain-bootstrap:issue}

\subsection{卵が先か鶏が先か問題}

\subsection{新機能の追加}

\subsection{依存フレームワークの対応}

\subsection{性能の低下}

%%% Local Variables:
%%% mode: japanese-latex
%%% TeX-master: "../bthesis"
%%% End:
