卒業論文要旨 - 2015年度 (平成27年度)
\begin{center}
\begin{large}
\begin{tabular}{|M{0.97\linewidth}|}
    \hline
    Self-host化によるSwiftコンパイラのソースコード可読性の向上\\
    \hline
\end{tabular}
\end{large}
\end{center}

~ \\

2015年12月にApple社が中心となって開発しているプログラミング言語Swiftがオープンソース化された。
オープンソースプロジェクトでは開発者の自発的な実装やレビューによって開発が進められるため、プロジェクトを活発化するためには開発者のモチベーションを向上することが重要となる。
その点において、SwiftコンパイラはObjective-Cの代替言語として発表されたために開発者から多くの関心を集めているが、特に開発コストを決める主要因である可読性にはまだ改善の余地があり、今後のSwiftコンパイラではその可読性を向上していくことがプロジェクトを活発化するために必要となる。

そこで本研究では、特定のプログラムをコンパイルするために必要なSwiftコンパイラ中のソースコードの行数をSwiftコンパイラの可読性を表す実行部分LOCという値で表し、この値を減少させることを目的として、コンパイル対象となる言語自体でそのコンパイラを記述するSelf-host化を用いた。

コンパイラを記述する言語とその対象言語が同じになれば開発者はより少ない知識でコンパイラのコードを読むことができる上、初期のコンパイラにおいては、それを記述している言語よりもそのコンパイル対象となっている言語のほうが必ず後発のものであるため、多くの場合により表現力が高く、ソフトウェアの複雑性を低くできる可能性が高い。
しかし、Swiftにおいては現行のコンパイラを記述しているC++も様々な特徴を持つ高級汎用言語であるため、その複雑性における優位性は明らかではない。

本研究ではSwiftで記述されたSwiftコンパイラの構文解析器を実装し、現行のSwiftコンパイラの構文解析器とその可読性比較した結果、平均して10.47\%の実行部分LOCの減少を達成することができた。
ただし、SwiftのSelf-host化による可読性の向上にはSwift独自の構文が関わっているため、他の言語においては既に高級言語で記述されているコンパイラをSelf-host化したとしても、同じ結果が得られるとは限らない。

~ \\
キーワード:\\
\underline{1. コンパイラ},
\underline{2. Self-host化},
\underline{3. プログラムの可読性},\\
\underline{4. プログラミング言語Swift}
\begin{flushright}
慶應義塾大学 環境情報学部\\
出水 厚輝
\end{flushright}
