卒業論文要旨 - 2015年度 (平成27年度)
~ \\
\begin{center}
\begin{Large}
\begin{tabular}{|c|} \hline
SwiftコンパイラのBootstrap化による可読性の向上
\\ \hline
\end{tabular}
\end{Large}
\end{center}
%~ \bigskip ~ \\

オープンソースとなったソフトウェアにおいては、その開発に関わるプログラマの増加とそれに伴うプログラムの修正や機能追加の増加によって、そのソースコードの可読性が拡張やそれに対するレビューの容易さに影響し、プロジェクト自体の成否に大きく関わる可能性がある。

2015年12月にオープンソースになったApple社が中心となって開発しているプログラミング言語Swiftもそうした可能性の分岐点に立つソフトウェアの1つである。
現在Swiftコンパイラの可読性は既存コードのコーディングスタイルへの習慣的な追従とレビューの徹底によって保たれているが、この形だけでは新しく追加されたコードの増加やプロジェクトメンバーの交代などによってその可読性が保てなくなる可能性が高い。

一方で、Swiftでは行われていないものの、現在利用されている多くの高級な汎用プログラミング言語では、コンパイル対象となる言語自体でそのコンパイラを記述するBootstrapがよく行われている。
Bootstrapを行うことによるメリットはいくつかあるが、たびたびモチベーションとしてあげられるのは、初期のコンパイラにおいてはそれを記述している言語よりも、そのコンパイラの対象となっている言語のほうが後発のものとなるため、より表現力が高く、可読性においてもより高い水準となることが多いという点である。

Swiftの開発者チームではC++で記述されたそのコンパイラをSwiftで書き直すよりも現在検討されている仕様の追加などを優先しており、Bootstrapについて具体的な議論を開始するまでには至っていない。
しかし、オープンソースとなったことで、Swiftにおいてより高い可読性が要求されるようになる可能性が上がっている以上、Bootstrapが可読性の向上に対して十分な成果を与えるのであれば、それを見当する価値は十分にあると考えられる。

そこで本研究では、Swiftで記述したSwiftコンパイラを実装し、その構文解析器のソースコードの行数を現行のSwiftコンパイラ中の構文解析器と比較することで、SwiftがBootstrapを行うことによってその可読性を向上させられることを検証した。
検証の結果として、本論文ではBootstrapを行うことによってコンパイラの可読性が向上する可能性が十分にあることを示しているが、同時にBootstrapがコンパイラに対して与える他の影響を鑑みると、実際の適用に際してはより慎重にならなければいけない点があるということがわかった。

~ \\
キーワード:\\
\underline{1. コンパイラ・ブートストラップ},
\underline{2. 構文解析},
\underline{3. 構文解析器の実装},\\
\underline{4. プログラミング言語Swift}
\begin{flushright}
慶應義塾大学 環境情報学部\\
~ \\
\begin{Large}
出水 厚輝
\end{Large}
\end{flushright}
