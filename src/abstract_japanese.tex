卒業論文要旨 - 2015年度 (平成27年度)
\begin{center}
\begin{Large}
\begin{tabular}{|c|} \hline
SwiftコンパイラのSelf-host化による可読性の向上
\\ \hline
\end{tabular}
\end{Large}
\end{center}
%~ \bigskip ~ \\

オープンソースとなったソフトウェアにおいては、その開発に関わるプログラマの増加とそれに伴うプログラムの修正や機能追加の増加によって、そのソースコードの可読性が拡張やそれに対するレビューの容易さに影響し、プロジェクト自体の成否に大きく関わる場合がある。

2015年12月にオープンソースになったApple社が中心となって開発しているプログラミング言語Swiftもそうした可能性の分岐点に立つソフトウェアの1つである。
現在Swiftコンパイラの可読性は既存コードのコーディングスタイルへの習慣的な追従とレビューの徹底によって保たれているが、この形だけでは新しいコードの増加やプロジェクトメンバーの交代などによってその可読性が保てなくなる可能性が高い。

一方で、Swiftでは行われていないものの、現在利用されている多くの高級な汎用プログラミング言語では、コンパイル対象となる言語自体でそのコンパイラを記述するSelf-host化がよく行われている。
Self-host化を行うことによるメリットはいくつかあるが、たびたびモチベーションとしてあげられるのは、その可読性における優位点である。
初期のコンパイラにおいては、それを記述している言語よりもそのコンパイル対象となっている言語のほうが必ず後発のものであるため、多くの場合により表現力が高く、可読性においてもより高い水準となるためである。

しかし、Swiftの開発者チームではC++で記述されたそのコンパイラをSwiftで書き直すよりも現在検討されている仕様の追加などを優先しており、Self-host化について具体的な議論を開始するまでには至っていない。
ただ、オープンソースとなったことでSwiftにおいてより高い可読性が要求されるようになってきている以上、Self-host化が可読性の向上に対して十分な成果を与えるのであれば、それを見当する価値は十分にあると考えられる。

そこで本研究では、Swiftで記述したSwiftコンパイラを実装し、Self-host化によってSwiftコンパイラの可読性を向上させられることを検証した。
検証には、実装したSwiftコンパイラの構文解析器を構成するソースコードの行数と現行のSwiftコンパイラにおいて同様の機能を担う箇所のソースコードの行数を比較した結果を用いている。
検証の結果として、本論文ではSwiftコンパイラのSelf-host化によってその可読性が向上する可能性が十分にあることを示した。
だだし、同時に行ったSelf-host化に伴うデメリットを考慮した検証により、Self-host化がコンパイラに対して与える他の影響を鑑みると、実際の適用に際してはより慎重にならなければいけない点があるということもわかった。

~ \\
キーワード:\\
\underline{1. コンパイラ},
\underline{2. Self-host化},
\underline{3. プログラムの可読性},\\
\underline{4. プログラミング言語Swift}
\begin{flushright}
慶應義塾大学 環境情報学部\\
出水 厚輝
\end{flushright}
