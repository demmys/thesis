卒業論文要旨 - 2015年度 (平成27年度)
\begin{center}
\begin{large}
    \begin{tabular}{|M{0.97\linewidth}|} \hline
Self-host化によるSwiftコンパイラのソースコード可読性の向上
\\ \hline
\end{tabular}
\end{large}
\end{center}
%~ \bigskip ~ \\

オープンソースとなったソフトウェアにおいては、その開発に関わるプログラマの増加とそれに伴うプログラムの修正や機能追加の増加によって、そのソースコードの可読性が拡張やそれに対するレビューの容易さに影響し、プロジェクト自体の成否に大きく関わる場合がある。

2015年12月にオープンソースとなったApple社が中心となって開発しているプログラミング言語Swiftもそうした可能性の分岐点に立つソフトウェアの1つである。
現在Swiftコンパイラの可読性は既存コードのコーディングスタイルへの習慣的な追従とレビューの徹底によって保たれているが、この形だけでは新しいコードの増加やプロジェクトメンバーの交代などによってその可読性が保てなくなる可能性が高い。

一方で、Swiftでは行われていないものの、現在利用されている多くの高級な汎用プログラミング言語では、コンパイル対象となる言語自体でそのコンパイラを記述するSelf-host化がよく行われている。
Self-host化を行うことによるメリットはいくつかあるが、たびたびモチベーションとしてあげられるのは、その可読性における優位点である。
コンパイラを記述する言語とその対象言語が同じになれば開発者はより少ない知識でコンパイラのコードを読むことができる上、初期のコンパイラにおいては、それを記述している言語よりもそのコンパイル対象となっている言語のほうが必ず後発のものであるため、多くの場合により表現力が高く、可読性においてもより高い水準となるからである。

そこで本研究では、Swiftで記述されたSwiftコンパイラの構文解析器を実装し、現行のSwiftコンパイラの構文解析器とそのソースコードの行数を多面的に比較することでSelf-host化がSwiftコンパイラに与える影響についての検証を行った。
検証の結果として、本論文ではSwiftコンパイラのSelf-host化によってその可読性が向上する可能性が十分にあることを示している。
だだし、同時に行ったSelf-host化に伴うデメリットの考察により、Self-host化がコンパイラに対して与える他の影響を鑑みると、実際の適用に際してはより慎重にならなければいけないということもわかった。

~ \\
キーワード:\\
\underline{1. コンパイラ},
\underline{2. Self-host化},
\underline{3. プログラムの可読性},\\
\underline{4. プログラミング言語Swift}
\begin{flushright}
慶應義塾大学 環境情報学部\\
出水 厚輝
\end{flushright}
