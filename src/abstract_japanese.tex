卒業論文要旨 - 2015年度 (平成27年度)
\begin{center}
\begin{large}
\begin{tabular}{|M{0.97\linewidth}|}
    \hline
    Self-host化によるSwiftコンパイラのソースコード可読性の向上\\
    \hline
\end{tabular}
\end{large}
\end{center}

~ \\

Swiftがオープンソースになった

オープンソースとなったソフトウェアにおいては、その開発に関わるプログラマの増加とそれに伴うプログラムの修正や機能追加によって、そのソースコードの可読性が拡張性やレビューの容易さに影響し、プロジェクト自体の成否に大きく関わる場合がある。

2015年12月にオープンソースとなったApple社が中心となって開発しているプログラミング言語Swiftもそうした可能性の分岐点に立つソフトウェアの1つである。
現在Swiftコンパイラの可読性は既存コードのコーディングスタイルへの習慣的な追従とレビューの徹底によって保たれているが、これらの方法のみでは新しいコードの増加やプロジェクトメンバーの交代などによってその可読性が保てなくなる可能性が高い。

一方で、Swiftでは行われていないものの、現在利用されている多くの高級な汎用プログラミング言語では、コンパイル対象となる言語自体でそのコンパイラを記述するSelf-host化がよく行われている。
Self-host化を行うことによるメリットはいくつかあるが、たびたびモチベーションとしてあげられるのは、その可読性における優位点である。
コンパイラを記述する言語とその対象言語が同じになれば開発者はより少ない知識でコンパイラのコードを読むことができる上、初期のコンパイラにおいては、それを記述している言語よりもそのコンパイル対象となっている言語のほうが必ず後発のものであるため、多くの場合により表現力が高く、ソフトウェアの複雑性を低くできる可能性が高いからである。
しかし、Swiftにおいては現行のコンパイラを記述しているC++も様々な特徴を持つ高級汎用言語であるため、その複雑性における優位性は明らかではない。

そこで本研究では、Swiftで記述されたSwiftコンパイラの構文解析器を実装し、現行のSwiftコンパイラの構文解析器とそのソースコードの行数を幾つかの部分に分けて比較することで、Self-host化によってSwiftコンパイラの可読性が向上することを検証した。
ただし、SwiftのSelf-host化による可読性の向上にはSwift独自の構文が関わっているため、他の言語においては既に高級言語で記述されているコンパイラをSelf-host化したとしても、同じ結果が得られるとは限らない。

~ \\
キーワード:\\
\underline{1. コンパイラ},
\underline{2. Self-host化},
\underline{3. プログラムの可読性},\\
\underline{4. プログラミング言語Swift}
\begin{flushright}
慶應義塾大学 環境情報学部\\
出水 厚輝
\end{flushright}
