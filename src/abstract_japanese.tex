卒業論文要旨 - 2015年度 (平成27年度)
~ \\
\begin{center}
\begin{Large}
\begin{tabular}{|c|} \hline
Bootstrapに向けたSwiftによるSwift構文解析器の設計と実装
\\ \hline
\end{tabular}
\end{Large}
\end{center}
~  \\
%~ \bigskip ~ \\

現在利用されている多くの高級な汎用プログラミング言語では、コンパイル対象となる言語自体でそのコンパイラを記述するBootstrapが行われている。
Bootstrapを行うことによるメリットはいくつかあるが、度々モチベーションとしてあげられるのは、現存するプログラミング言語よりもBootstrapを行おうと考えているプログラミング言語のほうが後発のものであるため、より表現力が高く開発しやすいという点である。

しかし、近年開発されている汎用プログラミング言語に至っては、その言語自体だけでなく最初にコンパイラを記述する言語も高級なものとなっており、対象のコンパイラを記述する上でどちらの方がより高い表現力や性能を持つかを簡単に判断することはできなくなってきている。

Apple社が中心となって開発しているプログラミング言語Swiftもそのメリットとデメリットを明確に評価することができず、Bootstrapするべきか否かの判断を下せていない汎用プログラミング言語の1つである。
現在最も有名なSwiftのコンパイラ実装はC++で記述されており、コンパイラの核となる構文解析においてもC++の特徴的な機能を駆使して、より低級な言語ではボイラープレートとなるコードを排除している。
Swiftはその可読性の高さと実行速度の速さを謳った言語であるが、その性能がSwiftコンパイラという大規模なソフトウェアにおいてC++を相手としても通用するものであるかどうかを形式的に議論することは容易ではない。

そこで本研究では、Swiftで記述したSwiftの構文解析器を実装し、その性能とソースコードの行数を現行のSwiftコンパイラ中の構文解析器と比較することで、SwiftがBootstrapを行うための判断材料を収集・考察する。
本論文では、Swiftで構文解析器を書き換えることによって可読性につながりうるソースコードの行数の削減は実現できるが、性能の面においては未だSwift自体が充分な性能を持っていない可能性があることを示し、その結果からSwiftがBootstrapを行うならば必要になるであろうステップについて考察を行っている。

~ \\
キーワード:\\
\underline{1. コンパイラ・ブートストラップ},
\underline{2. 構文解析},
\underline{3. 構文解析器の実装},\\
\underline{4. プログラミング言語Swift}
\begin{flushright}
慶應義塾大学 環境情報学部\\
~ \\
\begin{Large}
出水 厚輝
\end{Large}
\end{flushright}
